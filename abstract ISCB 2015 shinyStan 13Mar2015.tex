%%%%%%%%%%%%%%%%%%%%%%%%%%%%%%%%%%%%%%%%%
% NIH Grant Proposal for the Specific Aims and Research Plan Sections
% LaTeX Template
% Version 1.0 (21/10/13)
%
% This template has been downloaded from:
% http://www.LaTeXTemplates.com
%
% Original author:
% Erick Tatro (erickttr@gmail.com) with modifications by:
% Vel (vel@latextemplates.com)
% Michael ma2196@columbia.edu
% with assistance from Jonah G.
% Adapted from:
% J. Hrabe (http://www.magalien.com/public/nih_grants_in_latex.html)
%
% License:
% CC BY-NC-SA 3.0 (http://creativecommons.org/licenses/by-nc-sa/3.0/)
%
%%%%%%%%%%%%%%%%%%%%%%%%%%%%%%%%%%%%%%%%%

%----------------------------------------------------------------------------------------
%	PACKAGES AND OTHER DOCUMENT CONFIGURATIONS
%----------------------------------------------------------------------------------------

\documentclass[11pt,notitlepage]{article}

% A note on fonts: As of 2013, NIH allows Georgia, Arial, Helvetica, and Palatino Linotype. LaTeX doesn't have Georgia or Arial built in; you can try to come up with your own solution if you wish to use those fonts. Here, Palatino & Helvetica are available, leave the font you want to use uncommented while commenting out the other one.

\usepackage[super]{natbib}
\usepackage{palatino} % Palatino font
%\usepackage{helvet} % Helvetica font
\renewcommand*\familydefault{\sfdefault} % Use the sans serif version of the font
\usepackage[T1]{fontenc}
\linespread{1.05} % A little extra line spread is better for the Palatino font
\usepackage{hyperref} % to allow hyperlinks to websites on the internet
\usepackage[hypcap]{caption} % to point to the top of the image
\usepackage{lipsum} % Used for inserting dummy 'Lorem ipsum' text into the template
\usepackage{amsfonts, amsmath, amsthm, amssymb} % For math fonts, symbols and environments
\usepackage{graphicx} % Required for including images
\usepackage{booktabs} % Top and bottom rules for table
\usepackage{wrapfig} % Allows in-line images
\usepackage[labelfont=footnotesize]{caption} % Make figure numbering in captions bold
\usepackage[top=0.5in,bottom=0.5in,left=0.5in,right=0.5in]{geometry} % Reduce the size of the margin
\pagestyle{empty} % Remove page numbers

\hyphenation{ionto-pho-re-tic iso-tro-pic fortran} % Specifies custom hyphenation points for words or words that shouldn't be hyphenated at all

  
  % to reduce white space between PARAGRAPHS
\setlength{\parskip}{0pt}
\setlength{\parsep}{0pt}

  % additional parameters
%\setlength{\headsep}{0pt}
%\setlength{\topskip}{0pt}
%\setlength{\topmargin}{0pt}
%\setlength{\topsep}{0pt}
%\setlength{\partopsep}{0pt}

  % to reduce white space around figures
% \setlength{\textfloatsep}{0pt plus 0pt minus 0pt}

  % to reduce white space between SECTIONS
\usepackage[compact]{titlesec}
\titlespacing{\part}{0pt}{5pt}{4pt}
%\titlespacing{\subsection}{0pt}{*0}{*0}
%\titlespacing{\subsubsection}{0pt}{*0}{*0}
%\titlespacing{\subparagraph}{0pt}{*0}{*0}
\titlespacing*{\subparagraph} {\parindent}{1ex plus 1ex minus .2ex}{0.5em}


\begin{document}

\section*{Conference:}

Third Annual Conference of the International Society for Clinical Biostatistics, Utrecht Aug 22nd 2015. I will attend this conference, but maybe another (Bayesian) meeting might be more suitable?

\href{http://www.iscb.info/ISCB2015.html}{Conference Website}

\href{http://www.iscb2015.info/abstract-submission}{Call for Abstracts}



\section*{Title:}

Posterior exploration and predictive checking for Bayesian hierarchical models via interactive graphical analysis (140 characters max)
\section*{Abstract}
(1,750 characters max or 1,250 with one illustration) \textbf{One illustration (simple photograph/table/figure) may be uploaded for inclusion in the abstract. The maximum length of the abstract then becomes 1,250 characters, including spaces.}

\section*{Background:}
Exploratory and confirmatory data analyses complement each other in the comparison of data to implicit or explicit statistical models, but in practice can be cumbersome to realize and interpret in the face of complex hierarchical models and modern Markov chain Monte Carlo (MCMC) algorithms (e.g. Hamiltonian Monte Carlo). The availability of drastically increased computer power and advanced graphical software routines allows novel approaches to integrate exploratory and confirmatory analysis for complex hierarchical models.
  
\section*{Materials and methods:}
We developed shinyStan, an open source software package for graphical exploratory and confirmatory analysis of Bayesian models. shinyStan uses the Shiny web application framework and is coded in R, CSS, HTML, and JavaScript.  
\section*{Results:}

Computation of graphics and aggregate statistics for large MCMC data limited the responsiveness of our application to the user and we worked to find the best possible balance between startup and responsive computation. 3D visualization of asymmetric distributions of prior and posterior statistics on the log scale against the log-likelihood offer new insight into model misspecifications. For instance, certain surface patterns shown by these 3D plots motivates the use of non-centered parametrizations to drastically improve the ratio of effective sample size to total posterior draws in Hamiltonian MCMC.
\section*{Conclusions:}

Interactive and 3D graphical exploration of hierarchical models may provide novel means to detect model misspecification in the context of hierarchical Bayesian inference. shinyStan provides new tools detecting and demonstrating unexpected deviations from implicit model assumptions as well as an interface for general posterior analysis.

%----------------------------------------------------------------------------------------
%	BIBLIOGRAPHY
%----------------------------------------------------------------------------------------

\newpage

\bibliography{K01_bibliography_24Feb15} % Use the NIHbibliography.bib file for the reference list; the file name cannot contain spaces
\bibliographystyle{nihunsrt} % Use the custom nihunsrt bibliography style included with the template

%----------------------------------------------------------------------------------------

\end{document} 
